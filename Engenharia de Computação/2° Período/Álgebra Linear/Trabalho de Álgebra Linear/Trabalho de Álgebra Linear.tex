\documentclass[11pt,a4paper]{article}
\usepackage[margin=.5in]{geometry}
\usepackage[utf8]{inputenc}
\usepackage[portuguese]{babel}
\usepackage[T1]{fontenc}
\usepackage{amsmath}
\usepackage{amsfonts}
\usepackage{amssymb}
\usepackage{makeidx}
\usepackage{graphicx}
\usepackage{mathtools}
\usepackage{lmodern}
\usepackage{esvect}
\usepackage{xcolor}
\usepackage{lipsum}
\usepackage{setspace}
\newcommand\tab[1][1.835cm]{\hspace*{#1}}
\newcommand\taba[1][2.55cm]{\hspace*{#1}}
\newcommand\tabb[1][2.2cm]{\hspace*{#1}}
\newcommand\tabc[1][3cm]{\hspace*{#1}}
\author{Luis Alexandre Ferreira Bueno, Luiz Filipe de Jesus, Nicolas Timoteu Cuerbas e Vitor Bruno de Oliveira Barth}
\title{Trabalho de Álgebra Linear }

% End of Documento Declaration

\begin{document}

% Título
\begin{center}
\textbf{Trabalho de Álgebra Linear}
\end{center}

% Cabeçalho
\begin{flushleft}

\textbf{Professora: }Aline Brum Seibel\linebreak

%Alunos
\textbf{Alunos: } Luis Alexandre Ferreira Bueno\linebreak 
\tab Luiz Filipe de Jesus\linebreak
\tab Nicolas Timoteu Cuerbas\linebreak
\tab Vitor Bruno de Oliveira Barth\linebreak
	
%Conteudo	
\textbf{Conteúdos:} Matrizes canônicas, transformações lineares, operadores lineares, autovalores, autovetores e polinômios característicos \linebreak

%
% DOCUMENTO
%

\textbf{1. Matrizes canônicas e transformações lineares}\linebreak
\\
\tab \textbf{I) Isomorfismo:} Seja $\phi: V \rightarrow W$ uma transformação linear. Para que ela exista, é necessário que sejam atendidas algumas condições: \linebreak
\tabb a) $V$ e $W$ precisam estar sobre o mesmo plano \linebreak
\tabb b) $\phi$ precisa ser bijetora, ou seja, todo elemento em W precisa ser necessáriamente e exclusivamente a imagem de um elemento em V\linebreak
\tabb c) $\phi$ deve obedecer as regras de soma e multiplicação por escalar: \linebreak
\taba $\vv{v_1}, \vv{v_2} \in V$, $\phi(\vv{v_1}), \phi(\vv{v_2}) \in W$ e $c \in \mathbb{R} $ \linebreak
\taba i) $\phi(\vv{v_1}) + \phi(\vv{v_2}) = \phi(\vv{v_1} + \vv{v_2}) $ \linebreak
\taba ii) $\phi(c*\vv{v_1}) = c*\phi(\vv{v_1}) $ \linebreak
\\
\tab \textbf{II) Homomorfismo:} Denomina-se uma transformação linear homomórfica a transformação  $T: V \rightarrow W$ em que a função T não é bijetora. \linebreak
\\
\tab \textbf{III) Base de uma Transformação:} Seja $V$ um espaço vetorial finito onde $ n = dim(V)$ e $\beta = \{ \vv{v_1}, \vv{v_2}, ..., \vv{v_n} \}$ a base deste espaço. Sendo assim, posso escrever todos os vetores deste espaço através da combinação $\vv{v} = c_1\vv{v_1} + c_2\vv{v_2} + ... + c_n\vv{v_n}$, onde $c$ pertence, claro, ao mesmo plano que os o espaço vetorial. Neste caso, $\begin{pmatrix} c_1 \\ \vdots \\ c_n \end{pmatrix}$ é a matriz das coordenadas de $V$.\linebreak
\tabb Se tenho uma a transformação linear $T: V \rightarrow W$, assim como a base de $V$, consigo dizer onde estão todos os vetores de $T(V)$, pois segundo as regras c.i) e c.ii) podemos dizer que $T(\vv{v})=T(c_1\vv{v_1}+c_2\vv{v_2}+...+c_n\vv{v_n})=$ $T(c_1\vv{v_1})+T(c_2\vv{v_2})+...+T(c_n\vv{v_n})= $ $c_1T(\vv{v_1})+c_2T(\vv{v_2})+...$ $ + c_nT(\vv{v_n})$ \linebreak
\\
\tab \textbf{IV) Imagem e núcleo de uma Transformação:} Sendo $T: V \rightarrow W$, $Im(T) = \{ \vv{w} \in W | $ $ \vv{w} = T(\vv{v})$ para alguns $\vv{v} \in V\}$, ou seja, a Imagem de uma Transformação corresponde ao subespaço vetorial de W que possui elementos dados por $T(\vv{v})$. Sendo assim, logicamente $Im(T) \leq W$. \linebreak
\tab Já o núcleo de uma Transformação (denotado por $Ker(T)$) é dado por $Ker(T) = \{ \vv{v} \in V $ $ | T(\vv{v}) = 0\}$. Se a transformação é injetora, o núcleo é trivial, afinal $T(0)$ deve obrigatóriamente pertencer à $Im(T)$. Porém caso ela não seja injetora, o núcleo poderá ser um conjunto vetorial. Sendo assim, obrigatoriamente, $Ker(T)$ é um subespaço vetorial de $V$ \linebreak
\\
\tab \textbf{V) Exemplos:} \linebreak
\taba \linebreak 
\\
\textbf{2. Operadores lineares}\linebreak
\\
\textbf{3. Autovalores e autovetores}\linebreak
\\
\textbf{4. Polinômios característicos}\linebreak
\tab \linebreak 
\textbf{5. Exercícios}\linebreak
\tab 1) Verifique se as seguintes afirmações são verdadeiras ou falsas \linebreak
\taba (a) Qualquer operador linear em $V$ é tal que $V = Ker(T) \oplus Im(T)$ \linebreak
\taba (b) Se $T:P_2\rightarrow\mathbb{R}^2$ uma transformação linear definida por $T:(at^2+bt+c) = (a-b$ $+c,  2a +b-c)$, então $\vv{p}(t)=5t+5 \in Ker(T)$ \linebreak
\taba (c) Se $Ker(T)$ é gerado por três vetores $\vv{v_1}, \vv{v_2}, \vv{v_3}$, então a imagem de qualquer operador linear $T:\mathbb{R}^5\rightarrow\mathbb{R}^5$ tem dimensão 2\linebreak
\taba (d) A aplicação linear $T:M(2,2)\rightarrow\mathbb{R}$ definida por $T\begin{pmatrix}\begin{bmatrix} a & b \\c & d\end{bmatrix}\end{pmatrix} = $ $2a+c-d$ é uma transformação linear\linebreak
\taba (e) Existem transformações lineares $T:P_1 \rightarrow P_3$ sobrejetoras\linebreak
\\
\tab 2) Seja a transformação linear $T:\mathbb{R}^3 \rightarrow \mathbb{R}^2$ definida por $T(x,y,z) = (x-y+2z,4x+ $ $3y-z)$ determine: \linebreak
\taba(a) A matriz canônica de $T$ \linebreak
\taba(b) O núcleo de $T$, uma base e a dimensão \linebreak
\taba(c) A imagem de $T$, uma base e a dimensão \linebreak
\\
\tab 3) Determine a transformação linear que leva os vetores $\vv{c_1}, \vv{c_2}, \vv{c_3}$ nos vetores $\vv{w_1}=$ $(1,0,0)$, $\vv{w_2}=(3,1,0)$ e $\vv{w_3}=(1,2,4)$ respectivamente. e responda se esta transformação linear é um isomorfismo\linebreak
\\
\tab 4) Dada a matriz canônica $[T]= \begin{bmatrix}
1 & 0 & 0\\ 2 & 1 & 0\\ 3 & -2 & -1 \end{bmatrix}$ de um operador linear em $\mathbb{R}^3$, verifique se $T$ é um isomorfismo e justifique se $\vv{w}=(2,-1,0) \in Im(T)$ e se $\vv{u}= $ $(0,3,4) \in Ker(T)$?\linebreak



%
% BIBLIOGRAFIA
%
\textbf{Bibliografia: }BOLDRINI, Jose Luís. \textit{Álgebra Linear}. 3ª Edição. \linebreak
\taba COELHO, Flávio Uhoa. \textit{Um Curso de Álgebra Linear}. 2ª Edição.
\taba WIKIBOOKS. \textit{Álgebra Linear}. Edição de 24/01/2014. \linebreak
\end{flushleft}

\end{document}