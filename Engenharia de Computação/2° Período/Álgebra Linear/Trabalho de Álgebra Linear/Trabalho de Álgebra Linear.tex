\documentclass[11pt,a4paper]{article}
\usepackage[margin=.7in]{geometry}
\usepackage[utf8]{inputenc}
\usepackage[portuguese]{babel}
\usepackage[T1]{fontenc}
\usepackage{amsmath}
\usepackage{amsfonts}
\usepackage{amssymb}
\usepackage{makeidx}
\usepackage{graphicx}
\usepackage{mathtools}
\usepackage{lmodern}
\usepackage{esvect}
\usepackage{xcolor}
\usepackage{lipsum}
\usepackage{setspace}
\renewcommand{\baselinestretch}{1.0} 
\newcommand\tab[1][1.835cm]{\hspace*{#1}}
\author{Luis Alexandre Ferreira Bueno, Luiz Filipe de Jesus, Nicolas Timoteu Cuerbas e Vitor Bruno de Oliveira Barth}
\title{Trabalho de Álgebra Linear }

% End of Documento Declaration

\begin{document}

% Título
\begin{center}
\textbf{Trabalho Final de Álgebra Linear}
\end{center}

% Cabeçalho
\begin{flushleft}

\textbf{Professora: }Aline Brum Seibel\linebreak

%Alunos
\textbf{Alunos: } Luis Alexandre Ferreira Bueno\linebreak 
\tab Luiz Filipe de Jesus\linebreak
\tab Nicolas Timoteu Cuerbas\linebreak
\tab Vitor Bruno de Oliveira Barth\linebreak
	
%Conteudo	
\textbf{Conteúdos:} Autovalor, Autovetor, Operador Linear, Polinômio Característico e Transformação Linear\linebreak

%
% DOCUMENTO
%

\textbf{(a) Matrizes canônicas e transformações lineares}\linebreak
\linebreak
\textbf{(a).1 - Condição de Existência:}\linebreak
Julgando que $T: {\mathbb{R}}^n\rightarrow{\mathbb{R}^m} $ e $\vv{a}, \vv{b} \in {\mathbb{R}}$. Transformação Linear existirá se e somente se: \linebreak
\tab I) $T(\vv{a} + \vv{b}) = T(\vv{a}) + T(\vv{b})$\linebreak
\tab II) $T(c\vv{a}) = cT(\vv{a})$ \linebreak 

\textbf{(b) Definição de operadores lineares e exemplos}\linebreak

\textbf{(c) Definição de autovalor, autovetor e exemplos, como calcular}\linebreak
\\
\textbf{(d) Polinômimios característicos, definição e como calcular}\linebreak
\\
\textbf{(e) Exercícios}\linebreak
\\

%
% BIBLIOGRAFIA
%

\textbf{Bibliografia: }BOLDRINI, Jose Luís. \textit{Álgebra Linear}. 3ª Edição.
\end{flushleft}

\end{document}